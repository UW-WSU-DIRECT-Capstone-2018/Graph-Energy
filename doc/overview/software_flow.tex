% Main module software flow. Based on: 
%     Flowcharting techniques for easy maintenance
%     Original Author: Brent Longborough
%     Retreived from : http://www.texample.net/tikz/examples/flexible-flow-chart/
%     Retreived on: April 26, 2018
% =================================================
% -------------------------------------------------
% Set up a new layer for the debugging marks, and make sure it is on
% top
\pgfdeclarelayer{marx}
\pgfsetlayers{main,marx}
% A macro for marking coordinates (specific to the coordinate naming
% scheme used here). Swap the following 2 definitions to deactivate
% marks.
\providecommand{\cmark}[2][]{%
  \begin{pgfonlayer}{marx}
    \node [nmark] at (c#2#1) {#2};
  \end{pgfonlayer}{marx}
  } 
\providecommand{\cmark}[2][]{\relax} 

% =================================================
% -------------------------------------------------
% Start the picture
\begin{tikzpicture}[%
    >=triangle 60,              % Nice arrows; your taste may be different
    start chain=going below,    % General flow is top-to-bottom
    node distance=6mm and 60mm, % Global setup of box spacing
    every join/.style={norm},   % Default linetype for connecting boxes
    ]
% ------------------------------------------------- 
% A few box styles 
% <on chain> *and* <on grid> reduce the need for manual relative
% positioning of nodes
\tikzset{
  base/.style={draw, on chain, on grid, align=center, minimum height=4ex},
  proc/.style={base, rectangle, text width=10em},
  test/.style={base, diamond, aspect=3, text width=5em},
  term/.style={proc, rounded corners},
  inpt/.style={base, trapezium, trapezium left angle=120, trapezium right angle=60},
  % coord node style is used for placing corners of connecting lines
  coord/.style={coordinate, on chain, on grid, node distance=6mm and 25mm},
  % nmark node style is used for coordinate debugging marks
  nmark/.style={draw, cyan, circle, font={\sffamily\bfseries}},
  % -------------------------------------------------
  % Connector line styles for different parts of the diagram
}
% -------------------------------------------------
% Place the nodes
\node [term] (start) {Begin program};
\node [proc, join] (p1) {Parse data and input file};
\node [test, join] (t1) {Train?};
\node [proc] (p2) {Load trained net};
\node [proc, join] (p3) {Predict energy dist.};
\node [test, join] (t2) {Compare params?};
\node [proc] (p4) {Write output to file};
\node [proc, join] {Plot distribution(s)};
\node [term, join] {Termination};

% Side nodes
\node [inpt, left=of p1] (i1) {Data files};
\node [inpt] (i2) {Input file}; 

\node[inpt, left=of p4] (i3) {Output file};

\node [proc, right=of p2] (p5) {Generate net architecture};
\node [proc, join] (p6) {Train net};

% Invisible corner nodes
\node [coord, right=of i1] (c1) {}; % \cmark{1};
\node [coord, right=of i2] (c2) {}; % \cmark{2};
\node [coord, right=17.1em of t1] (c3) {}; % \cmark{3};
\node [coord, right=25em of t2] (c4) {}; % \cmark{4};
\node [coord, above=8.4em of c4] (c5) {}; % \cmark{5};

% Simple left/right connections
\draw [->] (i1.east) -- (p1);
\draw [->] (p6.west) -- (p3);
\draw [->] (p4.west) -- (i3);

% Simple top/bottom connections
\path (t1.south) to node [near start, xshift=1em] {$n$} (p2);
\draw [o->] (t1.south) -- (p2);

\path (t2.south) to node [near start, xshift=1em] {$n$} (p4);
\draw [o->] (t2.south) -- (p4);

% Corner paths
\draw[-] (i2.east) -- (c2);
\draw[-] (c2) -- (c1);

\path (t1.east) to node [near start, yshift=1em] {$y$} (c3);
\draw[*-] (t1.east) -- (c3);
\draw[->] (c3) -- (p5);

\path (t2.east) to node [near start, yshift=1em] {$y$} (c4);
\draw[*-] (t2.east) -- (c4);
\draw[-] (c4) -- (c5);
\draw[->] (c5) -- (p5.east);
\end{tikzpicture}
